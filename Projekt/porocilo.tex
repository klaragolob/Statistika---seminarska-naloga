\documentclass[a4paper]{article}

\usepackage[slovene]{babel}
\usepackage[utf8]{inputenc}
\usepackage[T1]{fontenc}
\usepackage{lmodern}
\usepackage{amsmath}
\usepackage{amssymb}
\usepackage{amsthm}
\usepackage{amsfonts}
\usepackage{mathtools}
\usepackage{enumitem}
\usepackage[table,xcdraw]{xcolor}
\usepackage[utf8]{inputenc} 
\usepackage[T1]{fontenc}
\usepackage{graphicx}


\begin{document}

\begin{titlepage}
    \centering
    \vfill
    \vfill
    \textbf{\large{SEMINARSKA NALOGA IZ STATISTIKE - POROČILO}}
    \vfill
    \textsc{\Large{Klara Golob}}
    \vfill\vfill
    \textsc{\large{UL FMF, Matematika - univerzitetni študij}}
    
     \vfill
    \large{Avgust 2020}
    
\end{titlepage}

%%%

\newpage

%%%%%%%%%%%%%%%%%%%%%%%%%%%%%%%%%%%%%%%%%%%%%%%%%%%%%%%%%%%%%%

\large{1. NALOGA}
\\
\\
V datoteki Kibergrad se nahajajo informacije o 43.886 družinah, ki stanujejo v mestu Kibergrad. Za vsako družino so zabeleženi naslednji podatki (ne boste potrebovali vseh):
\begin{itemize}
\item Tip družine (od 1 do 3)
\item Število članov družine
\item Število otrok v družini
\item Skupni dohodek družine
\item Mestna četrt, v kateri stanuje družina (od 1 do 4)
\item Stopnja izobrazbe vodje gospodinjstva (od 31 do 46)
\end{itemize}
Nalogo sem reševala s pomočjo programskega jezika R. Zraven je priložena datotek z imenom "naloga1.R", v kateri je postopek računanja.

\begin{enumerate}[label=(\alph*)]

\item Vzemite enostavni slučajni vzorec 200 družin in na njegovi podlagi ocenite povprečno število otrok na družino v Kibergradu. 
\begin{equation*}
\hat{\mu} = \frac{1}{n} \sum_{i=1}^{n} x_{i}
\end{equation*}
Povprečno število otrok na podlagi vzorca  = $1.025$

\item Ocenite standardno napako in postavite $95\%$ interval zaupanja. \\
Varianca:
\begin{equation*}
\hat{\sigma}^2 = \frac{1}{n-1} \sum_{i=1}^{N}(\hat{\mu}-x_{i})^2
\end{equation*}
Standardna napaka:
\begin{equation*}
\widehat{ se(\hat{\sigma}}) = \sqrt{\frac{\hat{\sigma}^2}{n} \left(1-\frac{n}{N}\right) }
\end{equation*}
Standardna napaka izračunana po zgornji formuli $ = 0.07579955$ \\
Interval zaupanja $= [0.8764356, 1.173564]$

\item Vzorčno povprečje in ocenjeno standardno napako primerjajte s populacijskim povprečjem in pravo standardno napako. Ali interval zaupanja iz prejšnje točke pokrije populacijsko povprečje? \\ \\
Vzorčno povprečje = 0.92 \\
Populacijsko povprečje =  0.9479333 \\
Ocena standardne napake vzorca =  0.07579955 \\
Standardna napaka populacije =  0 \\
Razlika vzorčnega in populacijskega povprečja = 0.07706672 \\
Razlika standardnih napak = 0.07579955 \\ \\
Interval zaupanja iz prejšne točke pokrije populacijsko povprečje, saj je $0.9479333 \in [0.8764356, 1.173564]$

\item Vzemite še 99 enostavnih slučajnih vzorcev in prav tako za vsakega določite $95\%$ interval zaupanja. Narišite intervale zaupanja, ki pripadajo tem 100 vzorcem. Koliko jih pokrije populacijsko povprečje? \\ \

95 intervalov zaupanja izmed 100ih pokrije populacijsko povprečje, ker se lahko vidi tudi na sliki.

\begin{figure}[h!]
\centering
\includegraphics[width=11cm]{Intervali_zaupanja200.png}
\caption{Intervali zaupanja za 100 enostavnih slučajnih vzorcev velikosti 200}
\end{figure}

\item Izračunajte standardni odklon vzorčnih povprečij za 100 prej dobljenih vzorcev. Primerjajte s pravo standardno napako za vzorec velikosti 200. \\ \\
Standardni odklon vzorčnih povprečij za 100 prej dobljenih vzorcev = 0.08253901 \\
Standardna napaka za vzorec velikosti 200 = 0.07579955 \\
Razlika = 0.006739461 

\item Izvedite prejšnji dve točki še na 100 vzorcih po 800 družin. Primerjajte in razložite razlike s teorijo vzorčenja. \\ \\

96 intervalov zaupanja izmed 100ih pokrije populacijsko povprečje.\\
Standardni odklon vzorčnih povprečij za 100 dobljenih vzorcev veliksti 800 = 0.04111221 \\
Standardna napaka za vzorec velikosti 800 = 0.03928324 \\
Razlika = 0.001828971\\
Rezultati so malo bolj natančni, saj smo izbrali večji vzorec in zato dobili boljše tezultate. 
\begin{figure}[h!]
\centering
\includegraphics[width=11cm]{Intervali_zaupanja800.png}
\caption{Intervali zaupanja za 100 enostavnih slučajnih vzorcev velikosti 800}
\end{figure}
\end{enumerate}

\large{2. NALOGA}
\\ \\
Populacijo sestavljajo trije stratumi, prva dva imata 1000, tretji pa ima 500 enot. Iz vsakega stratuma vzamemo enostavni slučajni vzorec desetih enot in vrednosti spremenljivke pridejo:\\ \\
1.stratum: 94 99 106 106 101 102 122 104 97 97 \\
2.stratum: 183 183 179 211 178 179 192 192 201 177 \\
3.stratum: 343 302 286 317 289 284 357 288 314 276\\ \\
Ocenite populacijsko povprečje in standardno napako vaše cenilke ter poiščite aproksimativni $95\%$ interval zaupanja. \\ \\
Velikost populacije: $N = 2500$ \\
Velikosti stratumov: $N_1 = 1000, N_2 = 1000, N_3 = 500$  \\
Velikost enostavnih slučajnih vzorcev, izbranih iz stratumov: $n_1 = n_2 = n_3 = n = 10$ \\
Velikosti deležev stratumov: $w_1 = 0.4, w_2 = 0.4, w_3 = 0.2 $ \\
Vzorčna povprečja stratumov: $\hat{\mu_1} = 98.3,  \hat{\mu_2} = 187.5,  \hat{\mu_3} = 305.6$ \\ 
\\
Ocena populacijskega povprečja:
\begin{equation*}
\overline{X} = \hat{\mu} = w_1\hat{\mu_1} + w_2 \hat{\mu_2} + w_3 \hat{\mu_3} =  0.4 \times 98.3 + 0.4 \times 187.5 + 0.2 \times 305.6 = 175.44
\end{equation*}
Ocena kvadrata standardne napake:
 \begin{equation*}
\widehat{ se^2 }= \sum_{i=1}^{3} x_i^2 \frac{N_i-n}{N_i-1}\frac{S_i}{n_1(n_1-1)},
\end{equation*}
kjer je 
 \begin{equation*}
S_i = \sum_{j=1}^{n} (X_{ij} - \hat{\mu_i})^2 
\end{equation*}
in $X_{i1}, X_{i2} \ldots X_{in}$ vrednosti spremenljivk na enotah vzorca i-tega stratuma.

\begin{multline*}
S_1 = (94 - 98.3)^2 + (99 - 98.3)^2 + (106 - 98.3)^2 + (106 - 98.3)^2 + (101 - 98.3)^2  \\
 + (102 - 98.3)^2 + (122 - 98.3)^2 + (104 - 98.3)^2 + (97 - 98.3)^2 \\
+ (97 - 98.3)^2 = 756.1000000000001
\end{multline*}
\begin{multline*}
S_2 = (183 - 187.5)^2 + (183 - 187.5)^2 + (179 - 187.5)^2 + (211 - 187.5)^2  \\ + (178 - 187.5)^2 + (179 - 187.5)^2 + (192 - 187.5)^2 + (192 - 187.5)^2 \\ + (201 - 187.5)^2 + (177 - 187.5)^2 = 1160.5
\end{multline*}
\begin{multline*}
S_3 = (343 - 305.6)^2 + (302 - 305.6)^2 + (286 - 305.6)^2 + (317 - 305.6)^2 \\ + (289 - 305.6)^2 + (284 - 305.6)^2 + (357 - 305.6)^2 + (288 - 305.6)^2 \\ + (314 - 305.6)^2 + (276 - 305.6)^2 = 6566.400000000001
\end{multline*}
\begin{multline*}
\widehat{ se^2 }= 0.4^2 \frac{1000-10}{1000} \frac{756.1}{10(10-1)} +  0.4^2 \frac{1000-10}{1000} \frac{1160.5}{10(10-1)} + \\ 0.2^2 \frac{500-10}{500} \frac{6566.4}{10(10-1)} = 182.099
\end{multline*}
Ocena standardne napake cenilke populacijskega povprečja:
 \begin{equation*}
\hat{se}= \sqrt{\widehat{ se^2 }} = \sqrt{182.099} = 13.5
\end{equation*}
Aproksimativni $95\%$ interval zaupanja:
 \begin{equation*}
[\hat{\mu} - Z_{\alpha}\times\hat{se}, \hat{\mu} + Z_{\alpha}\times\hat{se}] = [148.98, 201.9]
\end{equation*} 

\large{3. NALOGA} \\ \\
V datoteki ZarkiGama se nahajajo podatki o časovnih razmikih med 3.935 zaznanimi fotoni, torej medprihodni časi (v sekundah). \\ \\
Nalogo sem reševala s pomočjo programskega jezika Matlab, datoteka z imanom "naloga3.m" je priložena zraven. 
\begin{enumerate}[label=(\alph*)]
\item Naredite histogram medprihodnih časov. Se vam zdi, da je model s porazdelitvijo gama plavzibilen?
\begin{figure}[h!]
\centering
\includegraphics[width=10cm]{histogram3.png}
\caption{Histogram medprehodnih časov}
\label{fig:hist}
\end{figure}

Glede na dobljeni histogram, ki je prikazan na sliki \ref{fig:hist}, je gama porazdelitev primerna.

\item Ocenite parametra porazdelitve gama po metodi momentov in po metodi največjega verjetja. Primerjajte!

$$X  \sim \Gamma(\alpha, \lambda)$$
Po metodi momentov sta cenilki za $\alpha$ in $\lambda$ (formuli sta iz strani 263 in 264 v knjigi Rice J.A. Mathematical statistics and data analysis (3rd)):
$$\hat{\alpha} = \frac{\overline{X}^2}{\hat{\sigma}^2} \ \ \ \text{in} \ \ \ \hat{\lambda} = \frac{\overline{X}}{\hat{\sigma}^2} $$
$$\overline{X} = 79.93522$$
$$\hat{\sigma} = 79.45616$$
$$\hat{\alpha} = 1.0121 \ \ \  \text{in} \ \ \ \hat{\lambda} =  0.0127 $$

Po metodi najmanjših kvadratov, pa cenilki za $\alpha$ in $\lambda$ dobimo z naslednjima dvema izrazoma (iz strani 270 v knjigi Rice J.A. Mathematical statistics and data analysis (3rd)):
$$n\log{\tilde{\alpha}} - n\log{\overline{X}} + \sum_{i=1}^{n} \log{x_i} - n\frac{\Gamma'(\tilde{\alpha})}{\Gamma(\tilde{\alpha})}$$
$$\tilde{\lambda} = \frac{\tilde{\alpha}}{\overline{X}}$$
Naj bo $F(x) = \frac{\Gamma'(x)}{\Gamma(x)}$
Potem je 
$$n\log{\tilde{\alpha}} - n\log{\overline{X}} + \sum_{i=1}^{n} \log{x_i} - nF(\tilde{\alpha})$$

Enačbo lahko rešimo s programom Matlab z uporabo funkcije psi in dobimo:
$$\tilde{\alpha} = 1.0263 \ \ \ \text{in} \ \ \ \tilde{\lambda} = 0.0128$$
Dobljeni oceni po različnih metodah sta skoraj enaki. 

\item Ocenjeni porazdelitvi dorišite na histogram. Je videti razumno? \\ \\
Na sliki \ref{fig:por} je graf, kjer je z roza barvo prikazana porazdelitev po metodi momentov in z modro barvo porazdelitev po metodi največjega verjetja. Porazdelitvi se ujemata med sabo in s histogramom.
\begin{figure}[h!]
\centering
\includegraphics[width=10cm]{histogram_porazdelitve3.png}
\caption{Histogram prehodnih časov z ocenjenima porazdelitvama}
\label{fig:por}
\end{figure}

\item Histogram z dorisanima gostotama narišite še na logaritemski lestvici. Lestvico transformirajte le na abscisni osi, vendar pa ustrezno transformirajte tudi dorisani gostoti.

Histogram na logaritemski lestvici je prikazan na sliki \ref{fig:log}.

\begin{figure}[h!]
\centering
\includegraphics[width=9cm]{log_histogram3.png}
\caption{Histogram prehodnih časov na logaritemski lestvici}
\label{fig:log}
\end{figure}

\item Je porazdelitev medprihodnih časov videti konsistentna s Poissonovim modelom, po katerem so ti časi porazdeljeni eksponentno? \\ \\
Da, glede na graf bi porazdelitev medprehodnih časov lahko ustrezala tudi Poissonovi porazdelitvi. Prav tako smo prameter $\alpha$ v obeh primerih ocenili blizu 1, iz česar bi lahko rekli, da so medprehodni časi porazdeljeni Exp($\lambda$), za ustrezen $\lambda$.
\end{enumerate}

\large{4. NALOGA} \\ \\
Recimo, da opazimo eno vrednost statistične spremenljivke X, porazdeljene enakomerno na intervalu $[0, \theta]$. Preizkusimo ničelno domnevo $H_0 : \theta = 1$ proti alternativni domnevi $H_1: \theta=2$.

\begin{enumerate}[label=(\alph*)]
\item Poiščite preizkus, ki ima stopnjo tveganja $\alpha = 0$. Kolikšna je njegova moč? \\ \\

Če je $\alpha = 0$, potem zagotovo ne bomo zavrgli ničelne hipoteze. Za X ki je porazdeljen enakomerno na $[0,1]$ ($\theta=0$), velja:
$$ \alpha = P(X > 1 | H_{0}) = 0$$.
Moč testa je verjetnost zavrnitve ničelne hipoteze v primeru, ko je ta v resnici napačna:
$$P(X < 1 | H_{1}) = \int_{0}^{1} f_X(x) dx = \int_{0}^{1} \frac{1}{2}dx = \frac{1}{2} $$

\item Za $0 < t < 1$ si oglejte preizkus, ki ničelno domnevo zavrne pri $X \leqslant t$. Kolikšni
sta njegova stopnja tveganja in moč? \\ \\
Stopnja tveganja:
$$P(X \in [0,t] | H_{0}) = \int_{0}^{t} f_X(x) dx = \int_{0}^{t} dx = t $$
Moč testa:
$$P(X\in [t, 2] | H_{1}) = \int_{t}^{2} f_X(x)dx = \int_{t}^{2} \frac{1}{2} dx =1- \frac{t}{2} $$

\item Naj bo spet $0 < t < 1$. Kolikšni sta stopnja tveganja in moč preizkusa, ki
ničelno domnevo zavrne pri $X \geqslant 1-t$ ? \\ \\
Stopnja tveganja: 
$$P(X \in [1-t,1] | H_{0}) = \int_{1-t}^{1} f_X(x) dx = \int_{1-t}^{1} dx = t $$
Moč preizkusa:
\begin{multline*} P(X\in [0,1-t] | H_{1}) + P(X\in [1,2] | H_{1})= \int_{0}^{1-t} f_X(x)dx +  \int_{1}^{2} f_X(x)dx =  \\\int_{0}^{1-t} \frac{1}{2} dx +  \int_{1}^{2} \frac{1}{2} dx= \frac{1-t}{2} + 1 - \frac{1}{2} = 1- \frac{t}{2} $$
\end{multline*}
\item Poiščite še kakšen preizkus, ki ima enako stopnjo tveganja in moč kot tisti iz
prejšnje točke. \\ \\
Očitno je primer takšnega preizkusa, preizkus iz točke b). Preizkusa sta različna, imata pa enko stopnjo tveganja in moč preizkusa.

\item Določite območje zavrnitve pri preizkusu na podlagi razmerja verjetij v odvisnosti od predpisane maksimalne stopnje tveganja. Kdaj je ta preizkus eksakten? \\ \\
$$\Lambda = \frac{P(X|H_{0})}{P(X|H_{1})} = \frac{f_X(x|H_0)}{f_X(x|H_{1})} = \left\{
  \begin{array}{lr}
    2, & x \in [0,1]\\
    0, & x \in (1,2]
  \end{array}
\right.  $$
Če je $\alpha = 0$ bomo ničelno hipotezo zagotovo zavrnili. Če je $\alpha > 0$, potem ničelno hipotezo zavrnemo, ko je $X \geqslant c$, za nek $c<1$.
$$\alpha = P(X \geqslant c | H_{0}) = \int_{c}^{\infty} f_X(x) dx = \int_{c}^{1} dx =1-c $$
Zato je $c = 1-\alpha$, kar pomeni, da je območje zavrnitve:
$$X \geqslant 1-\alpha$$

\item Kaj se zgodi s preizkusom na podlagi razmerja verjetij, če ničelno in alternativno domnevo zamenjamo, torej preizkusimo $H_{0}$ : $\theta = 2$ proti $H_1$ : $\theta = 1$ ?
$$\Lambda = \frac{P(X|H_{0})}{P(X|H_{1})} = \frac{f_X(x|H_0)}{f_X(x|H_{1})} = \left\{
  \begin{array}{lr}
    \frac{1}{2}, & x \in [0,1]\\
    \infty, & x \in (1,2]
  \end{array}
\right.  $$


\item Za situacijo iz prejšnje točke predlagajte še kakšen drug, eksakten preizkus in primerjajte moči obeh preizkusov.

\end{enumerate}

\large{5. NALOGA} \\ \\
Naj bosta $X$ in $Y$ slučajni spremenljivki z:
    \begin{equation*} \text{E}(X) = \mu_x, \ \ \  \text{E}(Y) = \mu_y, \end{equation*}
     \begin{equation*}\text{var}(X) = \sigma_x^2 \ \ \ \text{var}(Y) = \sigma_y^2, \end{equation*}
     \begin{equation*}\text{cov}(X,Y) = \sigma_{x,y} \end{equation*}.

Denimo da opazimo $X$ in želimo napovedati $Y$.
\begin{enumerate}[label=(\alph*)]
\item Poiščite napoved oblike $Y = \alpha + \beta X$, kjer $\alpha$ in $\beta$ izberemo tako, da je srednja
kvadratična napaka $ \text{E} [ ( Y - \hat{Y} ) ^2 ]$  minimalna. Matematični upanji, varianci in kovarianco poznamo
Pomagamo si z namigom:
\begin{align*} 
    \text{E} \left[ \left( Y - \hat{Y} \right) ^2 \right] = \left[ \text{E}(Y) - \text{E}(\hat{Y}) \right]^2 + \text{var}(Y - \hat{Y}).
\end{align*}
Poiskati moramo vrednosti za $\alpha$ in $\beta$, ki minimizirata desno stran zgornje enačbe. Ker sta oba člena enkao predznačena, sta pozitivna, lahko poiščemo $\alpha$ in $\beta$, ki minimizirata vsak člen posebej. Potem bo tudi vsota teh dveh plenov minimalna. Za prvi člen velja:
\begin{equation*}
\text{E}(\hat{Y}) = \alpha + \beta \text{E}(X) =  \alpha + \beta \mu_x,
\end{equation*}
\begin{equation*}
    \left[ \text{E}(Y) - \text{E}(\hat{Y}) \right]^2 = \left[ \mu_y - \alpha - \beta \mu_x \right] ^2 .
\end{equation*}
Ta bo najmanjša, ko bo $ \mu_y - \hat{\alpha} - \beta \cdot \mu_x = 0$. Iz tega sledi:
\begin{equation*}
\hat{\alpha} = \mu_y - \beta \mu_x.
\end{equation*} 
Za drugi člen pa velja:
\begin{multline*}
\text{var}(Y-\hat{Y}) = \text{var}(Y - \alpha - \beta X) = \text{var}(Y - \beta X) = \\ \text{var}(Y) - 2 \beta \text{cov}(X, Y) + \beta^2 \text{var}(X) = \sigma_y^2 - 2 \beta \sigma_{x,y} + \beta^2 \sigma_x^2.
\end{multline*}
Minimalno vrednost izraza izračunamo tako, da izraz odvajalmo po $\beta$ in odvod izenačimo z 0.
\begin{equation*}
    \frac{\partial}{\partial \beta} (\text{var}(Y - \hat{Y})) = - 2 \sigma_{x,y} + 2 \beta \sigma_x^2 = 0 
\end{equation*}
Tako je $ \hat{\beta} = \frac{\sigma_{x,y}}{\sigma_x^2} $.
\\
Vrednosti za $\alpha$ in $\beta$, ki minimizirata $\text{E} \left[ \left( Y - \hat{Y}) \right) ^2 \right]$ sta:

$$ \hat{\alpha} = \mu_y - \mu_x \frac{\sigma_{x,y}}{\sigma_x^2} \ \ \ \ \ \text{in} \ \ \ \ \ \hat{\beta} = \frac{\sigma_{x,y}}{\sigma_x^2} $$

\item Pokažite, da se pri tako izbranih koeficientih determinacijski koeficient (kvadrat korelacijskega koeficienta) izraža v obliki:
$$ r_{x,y}^2 = 1 - \frac{\text{var}(Y - \hat{Y})}{\text{var}(Y)}. $$
Po formuli za korelacijski koeficient velja:
$$ r_{x,y}^2 = \frac{\text{cov}(X,Y)^2}{\text{var}(X) \text{var}(Y)} = 1 - \frac{\text{var}(Y - \hat{Y})}{\text{var}(Y)} = \frac{\text{var}(Y) - \text{var}(Y - \hat{Y})}{\text{var}(Y)}. $$
Iz prejšnega primera uporabimo
$$ \text{var}(Y-\hat{Y}) = \sigma_y^2 - 2 \beta \sigma_{x,y} + \beta^2 \sigma_x^2,$$
$$ \beta = \frac{\sigma_{x,y}}{\sigma_x^2}.$$
\end{enumerate}
Vstavimo v enčbo in dobimo:
\begin{multline*}
 r_{x,y}^2 = \frac{\text{var}(Y) - \text{var}(Y - \hat{Y})}{\text{var}(Y)} = \frac{\sigma_y^2 - \sigma_y^2 + \frac{\sigma_{x,y}^2}{\sigma_x^2}}{\sigma_y^2} = \frac{\sigma_{x,y}^2}{\sigma_x^2 \sigma_y^2} =\frac{\text{cov}(X,Y)^2}{\text{var}(X) \text{var}(Y)}  
\end{multline*}
S tem smo dokazali, da se res koeficientih determinacijski koeficient izraža v obliki:
$$ r_{x,y}^2 = 1 - \frac{\text{var}(Y - \hat{Y})}{\text{var}(Y)}. $$

\end{document}